\section{Patterns 6 - Redegør for følgende concurrency mønstre}

\subsection{Fokuspunkter}

\begin{itemize}
	\item Parallel Loops.
	\item Parallel Tasks.
\end{itemize}

\subsubsection{Parallel loops}

Parallel loops pattern minder meget om et almindeligt loop. Der udføres samme operation hvor hvert element for et givet antal gange. Forskellen er dog at et almindeligt loop sker sekventielt, hvorimod et parallelt loop ofte udfører steps parallelt. Når man bruger parallelle loops er det derfor vigtigt at sikre sig at iterationerne ikke er afhængige af hinanden.

Eksempel på almindeligt for loop:

\begin{lstlisting}[caption=Normal for loop, label=code:normalLoop]
for (int i = 0; i < n; i++)
{
	//Do stuff...
}
\end{lstlisting}

Eksempel på parallelt for loop:

\begin{lstlisting}[caption=Parallel for loop,  label=code:paraLoop,
morekeywords={Parallel, For}]
Parallel.For (0, n, i =>
{
	//Do stuff...
});
\end{lstlisting}

\subsubsection{Parallel tasks}

En task er ikke en tråd! En task er en opgave som udføres \textbf{sekventielt}.\\

Vi kan bruge Task Parallel Library biblioteket til at køre disse tasks parallelt. TPL kan selv dynamisk scale graden af parallelisme sådan at det mest effektict udnytter de kerner den har til rådighed.

I kode udsnit~\ref{code:tasknormal} udføres en opgave på gammel sekventiel vis, mens listing\ref{code:taskparallel} viser bruges af \textit{Parallel.Invoke()}.

\begin{lstlisting}[
caption=Almindelig udførsel af opgave.,
label=code:tasknormal]
public void DoAll() {
	DoLeft();
	DoRight();
}
\end{lstlisting}

\begin{lstlisting}[
caption=Parallel udførsel af opgave/task.,
label=code:taskparallel,
morekeywords={Parallel, Invoke}]
public void DoAll() {
	Parallel.Invoke(DoLeft, DoRight);
}
\end{lstlisting}

